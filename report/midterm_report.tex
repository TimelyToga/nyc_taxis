\documentclass{article}
\usepackage{graphicx}
\usepackage{url}


%\usepackage{fullpage} % Makes the text margins smaller
\usepackage{graphicx} % To include figures
\usepackage{fancyvrb} % Includes the \VerbatimInput command to read in code files

% A very simple environment for writing pseudo-code
\newenvironment{pgm}{
  \begin{center}\begin{tabbing}
  xx \= xx \= xx \= xx \= xx \= xx \= xx \= xx \= xx \= xx \= xx \= \kill\>\+}{
  \end{tabbing}\end{center}}

\begin{document}
\begin{center}
	{\scshape \LARGE Examining New York City's Yellow Taxi Data Set \par}
	\vspace{0.3cm}
	{\scshape \large CS 516 Final Project - Midterm Report\par}
	\vspace{0.3cm}
	{ Ziyi \textsc{Wang}, Timothy \textsc{Blumberg}\par}
	\vspace{0.3cm}
	{ October 28, 2016\par}
	\vspace{1.5cm}
\end{center}

\section*{Abstract}

In our final project, we analyze NYC's {\textit very} public taxi dataset \cite{dataset} for interesting and surprising results. Our analysis has been principally done through queries on a SQL database, but because of the geographic nature of the data, we were forced to visualize from a very early stage in our project's formation. During this preliminary stage of developing our project, we have established an efficient workflow and found areas to engage in a more prolonged analysis during the remainder of the semester.

\section{The Data}
The dataset is extremely large (there is about $1.6$Gb of data produced every month at present date), so this creates many challenges as we attempt to gain insights from it. A powerful DBMS helps to cut down our query runtime considerably. The data is relatively clean given its size and complexity, and definitions for the coded portions of the data (such as the {\tt payment\_type} field) is given in the data dictionary \cite{dictionary}. The NYC Taxi \& Limousine Commission (TLC) collects and reports data for three different kinds of vehicles in NYC: yellow taxis, green taxis, and for-hire vehicles (FHV). Yellow taxis provide street-hailing service in Manhattan, Green taxis are designed to be useful when getting around in the boroughs of NYC, and the FHVs are available only through pre-arranging the pickup (i.e. cannot provide service that was not pre-arranged). For our project, we focus exclusively on the yellow taxis. 

Each row contains start and end time, pickup and drop-off coordinates, number of passengers (as reported by the driver), fare amount, tip amount, distance traveled and several others. We took a look at many of the fields individually as well as exploring relationships between several variables at a time.


\section{Interesting Pieces of our Dataset}

\vfill

\medskip
\bibliographystyle{unsrt}
\bibliography{midterm_report}


\end{document}
